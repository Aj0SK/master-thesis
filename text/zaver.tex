\chapter*{Conclusion}  % chapter* je necislovana kapitola
\addcontentsline{toc}{chapter}{Conclusion} % rucne pridanie do obsahu
\markboth{Conclusion}{Conclusion} % vyriesenie hlaviciek

The goal of this thesis was to study existing implementations of compressed
bit vector, explore their shortcomings and develop new implementation of bit
vector that mitigates some of these issues. We picked up one particular
representation of bit vector and come up with two ideas that try to enhance
it. We implemented both of these ideas and tested them as a part of \texttt{SDSL}
library that is regarded as one the most popular libraries implementing succinct
data structures.

The new decoding scheme that we introduced was very successful in artificial
but also practical testing. The most important result presented in this work
is the relative speedup of FM-index when \texttt{SDSL} bit vector is replaced
by our implementation. We measured about 20--30\% speedup of $\countOp$,
$\extractOp$ and $\locateOp$ methods thanks to our changes. This was observed
on various data types such as DNA sequences, source codes or protein sequences.

We developed two versions of hybrid encoding. The first is better suited for
sparse sequences with low frequency of ones and was meant to rival sparse bit
vector implementations on sequences with roughly 5\% of ones. This version indeed
worked quite well and when compared to our ordinary version of bit vector it was
able to deliver the same speed with roughly 5\% lower space usage.
The second version of hybrid encoding was found not to be practically competitive
in its current version due to additional memory accesses that are needed to answer
$\rank$ and $\select$ queries. 

With our new decoding scheme, future work could lead to discovering what tradeoffs
between space usage and speed can be achieved with longer blocks, e.g. 255 bit block.
Working with longer blocks is way slower because computers natively support 64-bit
integers. Our new method, however, enables us to quickly decompose problem into smaller
subproblems that could fit into a 64-bit integer.

%Na záver už len odporúčania k samotnej kapitole Záver v bakalárskej
%práci podľa smernice \cite{smernica}:  \glqq{}V závere je potrebné v
%stručnosti zhrnúť dosiahnuté výsledky vo vzťahu k stanoveným
%cieľom. Rozsah záveru je minimálne dve strany. Záver ako kapitola sa
%nečísluje.\grqq{}

%V informatických prácach niekedy býva záver kratší ako dve strany, ale
%stále by to mal byť rozumne dlhý text, v rozsahu aspoň jednej strany.
%Okrem dosiahnutých cieľov sa zvyknú rozoberať aj otvorené problémy a
%námety na ďalšiu prácu v oblasti.

%Abstrakt, úvod a záver práce obsahujú podobné informácie. Abstrakt je
%kratší text, ktorý má pomôcť čitateľovi sa rozhodnúť, či vôbec prácu
%chce čítať. Úvod má umožniť zorientovať sa v práci skôr než ju začne
%čítať a záver sumarizuje najdôležitejšie veci po tom, ako prácu
%prečítal, môže sa teda viac zamerať na detaily a využívať pojmy
%zavedené v práci.