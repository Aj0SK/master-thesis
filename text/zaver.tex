\chapter*{Conclusion}  % chapter* je necislovana kapitola
\addcontentsline{toc}{chapter}{Conclusion} % rucne pridanie do obsahu
\markboth{Conclusion}{Conclusion} % vyriesenie hlaviciek

The goal of this thesis was to explore representations of compressed
bit vector, develop an efficient implementation and compare its
performance with existing solutions. We have been able to come up with
two new encoding schemes and test them in artificial but also practical
settings.

One is based on the new ordering of the blocks along the same class. We
showed how this new ordering enables us to decode efficiently by reusing
the decoding solution for smaller sub-blocks. We showed that the decoding
routine can be very competitive with the previously existing methods in
artificial settings but also on real-life data. In the end, we demonstrated
its usability as a part of wavelet tree in FM-index.

%Na záver už len odporúčania k samotnej kapitole Záver v bakalárskej
%práci podľa smernice \cite{smernica}:  \glqq{}V závere je potrebné v
%stručnosti zhrnúť dosiahnuté výsledky vo vzťahu k stanoveným
%cieľom. Rozsah záveru je minimálne dve strany. Záver ako kapitola sa
%nečísluje.\grqq{}

%V informatických prácach niekedy býva záver kratší ako dve strany, ale
%stále by to mal byť rozumne dlhý text, v rozsahu aspoň jednej strany.
%Okrem dosiahnutých cieľov sa zvyknú rozoberať aj otvorené problémy a
%námety na ďalšiu prácu v oblasti.

%Abstrakt, úvod a záver práce obsahujú podobné informácie. Abstrakt je
%kratší text, ktorý má pomôcť čitateľovi sa rozhodnúť, či vôbec prácu
%chce čítať. Úvod má umožniť zorientovať sa v práci skôr než ju začne
%čítať a záver sumarizuje najdôležitejšie veci po tom, ako prácu
%prečítal, môže sa teda viac zamerať na detaily a využívať pojmy
%zavedené v práci.