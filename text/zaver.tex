\chapter*{Conclusion}  % chapter* je necislovana kapitola
\addcontentsline{toc}{chapter}{Conclusion} % rucne pridanie do obsahu
\markboth{Conclusion}{Conclusion} % vyriesenie hlaviciek

The goal of this thesis was to study existing implementations of compressed
bit vector, explore their shortcomings and develop new implementation of bit
vector that mitigates some of the existing problems. We picked up RRR -- one
particular representation of bit vector and identified places where this
representation and work with it can be improved. 

The biggest shortcoming was that were only two widely used techniques to
encode and decode block and both have significant disadvantages. Table method
provides very fast encoding and decoding but uses too much space for longer
blocks. On the other hand, on the fly decoding supports longer blocks but
trades-off quite a lot of performance for this. We addressed this problem by
proposing and implementing new decoding method based on the divide-and-conquer
approach that can be used to divide the process of decoding of the block to more
sub-problems that can be solved recursively or using one of the existing decoding
methods.

We implemented this idea and tested it as a part of \texttt{SDSL} library that is
regarded as one of the most popular libraries implementing succinct data structures.
The new method was very successful in artificial but also practical testing. The most
important result presented in this work is the relative speedup of FM-index when
\texttt{SDSL} bit vector is replaced by our implementation. We measured about 20--30\%
speedup of $\countOp$, $\extractOp$ and $\locateOp$ methods thanks to our changes. This
was observed on various data types such as DNA sequences, source codes or protein sequences.

The second idea was of hybrid encoding. The idea is not to encode some of the possibly rare
blocks. By doing this we waste some space on blocks that are not compressed but gain
on every other block by decreasing number of bits used for class of the block.

We developed two versions of hybrid encoding. The one-sided version is better
suited for sparse sequences with roughly 5\% of ones where RRR is often dominated by sparse
bit vector implementations. This version indeed worked quite well and when compared to our
ordinary version of bit vector it was able to deliver the same
speed with roughly 5\% lower space usage. The second, two-sided version of hybrid encoding
was found not to be practically competitive in its current version due to additional memory
accesses that are needed to answer $\rank$ and $\select$ queries. 

With our new decoding scheme, future work could lead to discovering what tradeoffs
between space usage and speed can be achieved with longer blocks, e.g., 255-bit block.
Working with longer blocks is way slower because computers natively support 64-bit
integers. Our new method, however, enables us to quickly decompose problem into smaller
subproblems that could fit into a 64-bit integer.

%Na záver už len odporúčania k samotnej kapitole Záver v bakalárskej
%práci podľa smernice \cite{smernica}:  \glqq{}V závere je potrebné v
%stručnosti zhrnúť dosiahnuté výsledky vo vzťahu k stanoveným
%cieľom. Rozsah záveru je minimálne dve strany. Záver ako kapitola sa
%nečísluje.\grqq{}

%V informatických prácach niekedy býva záver kratší ako dve strany, ale
%stále by to mal byť rozumne dlhý text, v rozsahu aspoň jednej strany.
%Okrem dosiahnutých cieľov sa zvyknú rozoberať aj otvorené problémy a
%námety na ďalšiu prácu v oblasti.

%Abstrakt, úvod a záver práce obsahujú podobné informácie. Abstrakt je
%kratší text, ktorý má pomôcť čitateľovi sa rozhodnúť, či vôbec prácu
%chce čítať. Úvod má umožniť zorientovať sa v práci skôr než ju začne
%čítať a záver sumarizuje najdôležitejšie veci po tom, ako prácu
%prečítal, môže sa teda viac zamerať na detaily a využívať pojmy
%zavedené v práci.