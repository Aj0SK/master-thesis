\chapter{Our work}
\label{kap:kap3}

In this chapter, we shall look into our new proposals to enhance the existing implementation of RRR.

\section{Block encoding}

As we already discussed in Section~\ref{section:compressed_bv}, there are to the best of our knowledge two
widely used algorithms to encode/decode the block. The problem with the table decoding algorithm is the space
it is using and the inability to support a bigger block size resulting from the huge table sizes for bigger
block length. On the other hand, the on-the-fly decoding can be used to support a bigger block size with the
downside being longer encoding and decoding times. We shall now propose new way of encoding and decoding the blocks.
The objective of the new method is to create an alternative to the previous methods that enables a longer block size
while not hurting the runtime so significantly.

The main idea of our solution is to break the problem of finding the order of the block $B$ along the class $c$ in a
divide-and-conquer manner. This will in turn enable us to use the original table method to solve the smaller subproblems.
To facilitate our solution, we need to alter the respective order of the blocks in the same class. The previous solutions
used the lexicographical ordering for the blocks that shared the same class. In our solution, we also use the number of
ones to identify the class of block. However, the ordering of the blocks along the class will be slightly altered.
Let us take two blocks $A$ and $B$ that share the same class. The original solution would compare these two blocks
by their lexicographical order and ordered them accordingly. We shall instead imagine splitting both blocks $A$ and $B$
into half, creating two smaller blocks $A_1$ and $A_2$ and $B_1$ and $B_2$ respectively. Now, the original block
with smaller number of ones in the first half is declared as smaller. If $A$ and $B$ have the same number of ones in the first
blocks~($A_1$ and $B_1$), the order will be given according to the original lexicographical ordering of the first blocks --
$A_1$ and $B_1$. If $A_1 = B_1$, the tie will be eventually broken by the order of the second blocks in the lexicographical
ordering. Note that if two blocks $A$ and $B$ share the class and they have equal first blocks, then their second blocks are
also in the same class so their order is given. The example of the ordering can be found on Figure~\ref{obr:lexicographicalVsUs}.

\begin{figure}
	\centerline{
        \begin{tabular}{l c}
            \small 0&   \tt 000 011\\
            \small 1&   \tt 000 101\\
            \small 2&   \tt 000 110\\
            \small 3&   \tt 001 001\\
            \small 4&   \tt 001 010\\
            \small 5&   \tt 001 100\\
            \small 6&   \tt 010 001\\
            \small 7&   \tt 010 010\\
        \end{tabular}
        \hspace{3em}
        \begin{tabular}{l c}
            \small 8&   \tt 010 100\\
            \small 9&   \tt 100 001\\
            \small 10&  \tt 100 010\\
            \small 11&  \tt 100 100\\
            \small 12&  \tt 011 000\\
            \small 13&  \tt 101 000\\
            \small 14&  \tt 110 000\\
        \end{tabular}
	}
	\caption[TODO]{
        Example of our new ordering for the block size of six and class two~($b=6$ and $c=2$).
        Note that this may resemble a lexicographical ordering, there are differences
        after row twelve.
    }
	\label{obr:lexicographicalVsUs}
\end{figure}

We shall now show how we use this new ordering and that it is convenient to encode and decode $B$ in a divide and conquer manner.

\paragraph{Encoding}

To explain how the process of encoding works, we will take the table solution for the block size $b_t=15$ as the underlying
scheme we use to solve the basic subproblem. With this subroutine, we shall obtain the solution for the block size of $b=30$.
Consider a block $B$ of length $b$. The first step is to count the number of ones in $B$ as this is the class of the block $c$.
After computing class, our next objective is to find out the sequence number of the first and second block, this can be easily
done using the table algorithm. This algorithm gives us for a two blocks $B_1$ and $B_2$ their respective class $c_1$ and $c_2$
but also offsets $o_1$ and $o_2$. Now we shall count the number of blocks preceding $B$. There are three types of blocks preceding
$B$:

\begin{itemize}
    \item Blocks with the same first sub-block but smaller second sub-block. \label{chapter3:encoding:1}
    \item Blocks with the first sub-block being from the same class as is $B_1$ but with smaller second block. \label{chapter3:encoding:2}
    \item Blocks with the first sub-block being from the smaller class as is $B_1$. \label{chapter3:encoding:3}
\end{itemize}

The number of items that satisfy property~\ref{chapter3:encoding:1} is $o_2$. The number of blocks with the
property~\ref{chapter3:encoding:2} is ${15\choose c_1}$. Finally, the number of items with property~\ref{chapter3:encoding:3}
is $\sum_{i=0}^{c_1-1} {15\choose i} {15\choose c-i}$. Identifying the number of blocks preceding $B$ gives us the offset
$o$ of the block $B$. Together with the class number $c$ we can encode $B$ standardly as a pair $(c, o)$.

\paragraph{Decoding}

Decoding is a process of obtaining block $B$ from pair $(c, o)$. Obtaining $c$ is trivial as we need only to count the
number of ones in $B$. Then, we start to obtain more and more information about the $B$. The first information is what
is the number of ones -- $c_1$ -- in the first block $B_1$. $c_1$ together with $c$ of course gives us a number of
ones in the second block $B_2$. First we will define $C_i$ to be a number of blocks of length $b$ with less or equal than
$i$ ones in the first block. We then compute the values of $C_i$ for every $i$ ($0\leq i\leq c$). Note that from definition
$C_i = \sum_{j=0}^{i} {15 \choose j} {15 \choose c-j}$.

\section{Hybrid encoding}