\chapter{Our work}
\label{kap:kap3}

In this chapter, we shall look into our new proposals to enhance the existing implementation of RRR.

\section{Block encoding}

As we already discussed in Section~\ref{section:compressed_bv}, there are two possible ways how to
decode the block. The problem with the table decoding version is the space it is using and the inability to
support a bigger block size. On the other hand, the on-the-fly decoding can be used to support a bigger
block size with the downside being longer encoding and decoding times. We shall now propose our new
way of encoding and decoding the block. The objective of the new method is to create an alternative to
the previous methods that enables a longer block size while not hurting the runtime so significantly.

To facilitate our solution, we need to alter the respective order of blocks in the same class. The
previous solutions used the lexicographical ordering for the blocks that shared the same class or
the number of ones in the block. In our solution, we also use the number of ones to identify 
the class of block. However, to compare two blocks $A$ and $B$, we at first split the blocks into
halves with $A$ being split into $A_1$ and $A_2$ and $B$ into $B_1$ and $B_2$. We shall for a moment
assume that $b$ -- the length of the blocks $A$ and $B$ -- is divisible by two. After splitting the blocks,
we at first compare based on the number of ones in the first halves ($A_1$ and $B_1$). As smaller, we
declare the block containing a lower number of ones in the first half. If the number of ones in the first
block matches in $A_1$ and $B_1$, we then decide based on the lexicographical order of $A_1$ and $B_1$.
If these two are equal, we proceed and compare $A$ and $B$ by the lexicographical order of $A_2$ and $B_2$.

\section{Hybrid encoding}