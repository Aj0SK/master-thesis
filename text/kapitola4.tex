\chapter{Implementation and benchmarking}
\label{kap:kap4}

\section{Implementation}

\section{Benchmarking}

Most of the FM-index implementations provide at least methods:

\begin{itemize}
	\item $\mathit{count}(P)$ counts the number of occurrences of $P$ in text $T$
	\item $\mathit{locate}(P)$ returns all positions of pattern $P$ in text $T$
	\item $\mathit{extract}(i, j)$ returns the subsequence $T[i..j]$
\end{itemize}

The reason that the $\mathit{extract}$ method is useful and non-trivial is that FM-index
does not store the original sequence $T$ -- at least not in an easily readable form.