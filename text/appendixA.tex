\chapter*{Appendix A: electronic attachment}
\addcontentsline{toc}{chapter}{Appendix A}

All the relevant source code is available in the electronic attachment to this work.
The code is divided into two parts. Benchmarking part is contained in the folder named
master-thesis and the fork of \texttt{SDSL} library that can be found in sdsl-lite folder.
Both of these can be also accessed on \url{https://github.com/Aj0SK/master-thesis} and
\url{https://github.com/Aj0SK/sdsl-lite}.

The first repository contains 4 experiments in directory \texttt{master-thesis/code}:
\begin{itemize}
    \item \texttt{experiment1} - microbenchmarking of various RRR implementations
    \item \texttt{experiment2} - \texttt{SDSL} and our benchmark of our new method
    \item \texttt{experiment3} - our benchmark for hybrid implementation
    \item \texttt{experiment4} - \texttt{SDSL} benchmark of our new method using FM-index
\end{itemize}
In all these folders, there are scripts to run the benchmarks. We used these also to obtain
all the results in this work. 

The second repository is a fork of \texttt{SDSL} library. The individual features are implemented
on separate branches:
\begin{itemize}
    \item \texttt{master} - non-modified version of master branch
    \item \texttt{benchmark\_original} - master branch containing minor changes
        to limit the number of benchmarked blocks, make the tests random but deterministic
    \item \texttt{rrr\_benchmark\_our} - branch with 31, 63 and 127 bit specialization and benchmarking changes
    \item \texttt{rrr\_benchmark\_hybrid} - branch with hybrid 31, 63 and 127 bit specialization and benchmarking changes
    \item \texttt{rrr\_vector\_31\_spec} - branch with minimum changes, implementing 31 bit specialization
    \item \texttt{rrr\_vector\_63\_127\_spec} - implementing 63 and 127 bit specialization (branched from \texttt{rrr\_vector\_31\_spec})
    \item \texttt{rrr\_vector\_hybrid} - hybrid implementation for block size 31, 63, 127
    \item \texttt{rrr\_vector\_hybrid\_valley} - special hybrid implementation with cut out classes around the center
\end{itemize}